%%%%%%%%%%%%%%%%%%%%%%%%%%%%%%%%%%%%%%%%%
% Daily Laboratory Book
% LaTeX Template
%
% This template has been downloaded from:
% http://www.latextemplates.com
%
% Original author:
% Frank Kuster (http://www.ctan.org/tex-archive/macros/latex/contrib/labbook/)
%
% Important note:
% This template requires the labbook.cls file to be in the same directory as the
% .tex file. The labbook.cls file provides the necessary structure to create the
% lab book.
%
% The \lipsum[#] commands throughout this template generate dummy text
% to fill the template out. These commands should all be removed when 
% writing lab book content.
%
% HOW TO USE THIS TEMPLATE 
% Each day in the lab consists of three main things:
%
% 1. LABDAY: The first thing to put is the \labday{} command with a date in 
% curly brackets, this will make a new page and put the date in big letters 
% at the top.
%
% 2. EXPERIMENT: Next you need to specify what experiment(s) you are 
% working on with an \experiment{} command with the experiment shorthand 
% in the curly brackets. The experiment shorthand is defined in the 
% 'DEFINITION OF EXPERIMENTS' section below, this means you can 
% say \experiment{pcr} and the actual text written to the PDF will be what 
% you set the 'pcr' experiment to be. If the experiment is a one off, you can 
% just write it in the bracket without creating a shorthand. Note: if you don't 
% want to have an experiment, just leave this out and it won't be printed.
%
% 3. CONTENT: Following the experiment is the content, i.e. what progress 
% you made on the experiment that day.
%
%%%%%%%%%%%%%%%%%%%%%%%%%%%%%%%%%%%%%%%%%

%----------------------------------------------------------------------------------------
%	PACKAGES AND OTHER DOCUMENT CONFIGURATIONS
%----------------------------------------------------------------------------------------

\documentclass[idxtotoc,hyperref,openany]{labbook} % 'openany' here removes the gap page between days, erase it to restore this gap; 'oneside' can also be added to remove the shift that odd pages have to the right for easier reading

\usepackage[ 
  backref=page,
  pdfpagelabels=true,
  plainpages=false,
  colorlinks=true,
  bookmarks=true,
  pdfview=FitB]{hyperref} % Required for the hyperlinks within the PDF
  
\usepackage{booktabs} % Required for the top and bottom rules in the table
\usepackage{float} % Required for specifying the exact location of a figure or table
\usepackage{graphicx} % Required for including images
\usepackage{lipsum} % Used for inserting dummy 'Lorem ipsum' text into the template

\newcommand{\HRule}{\rule{\linewidth}{0.5mm}} % Command to make the lines in the title page
\setlength\parindent{0pt} % Removes all indentation from paragraphs

\graphicspath{{../reports/figures/}} % Sets the path for all graphics, so only name is needed afterwards.

%----------------------------------------------------------------------------------------
%	DEFINITION OF EXPERIMENTS
%----------------------------------------------------------------------------------------

%\newexperiment{example}{This is an example experiment}
%\newexperiment{example2}{This is another example experiment}
\newexperiment{First steps}{}
\newexperiment{Data cleaning}{}
\newexperiment{table}{This shows a sample table}
%\newexperiment{shorthand}{Description of the experiment}

%---------------------------------------------------------------------------------------

\begin{document}

%----------------------------------------------------------------------------------------
%	TITLE PAGE
%----------------------------------------------------------------------------------------

\frontmatter % Use Roman numerals for page numbers
\title{
\begin{center}
\HRule \\[0.4cm]
{\Huge \bfseries Lab Book \\[0.5cm] \Large Machine Learning}\\[0.4cm] % Degree
\HRule \\[1.5cm]
\end{center}
}
\author{\Huge C\'ordoba Romero, Javier \\ \Huge Corroto Mart\'in, Juan Jos\'e \\ \Huge Guerrero del Pozo, \'Alvaro \\ \\[2cm]} % Your name and email address
\date{Beginning 10 October 2018} % Beginning date
\maketitle

\tableofcontents

\mainmatter % Use Arabic numerals for page numbers

%----------------------------------------------------------------------------------------
%	LAB BOOK CONTENTS
%----------------------------------------------------------------------------------------

% Blank template to use for new days:

%\labday{Day, Date Month Year}

%\experiment{}

%Text

%-----------------------------------------

%\experiment{}

%Text

%----------------------------------------------------------------------------------------

\labday{Wednesday, 10 October 2018}
\experiment{First steps}

Done by Juan Jos\'e Corroto, Javier C\'ordoba and \'Alvaro Guerrero\\

We've initialized the repo with \texttt{coookiecutter} directory structure and learnt the purpose of each directory.
We've also learnt how to work with \texttt{pandas}, read and write a csv and how to work with spyder.

%----------------------------------------------------------------------------------------

\experiment{Data cleaning}

We've picked the data of only the first day, we've done this by looking at the raw data and looking at the column \textit{"TimeStemp"}.

Then we've deleted the \textit{UUID, Version} and \textit{TimeStemp} columns because they were non-numerical values.

Then we've tried deleting the rows that had \textit{NaN} values, and we ended up deleting the whole dataframe. So, after scanning the dataframe, we have realized that the columns 
\begin{enumerate}
\item \textit{RotationVector\_cosThetaOver2\_MEAN}
\item \textit{RotationVector\_cosThetaOver2\_MEDIAN}
\item \textit{RotationVector\_cosThetaOver2\_MIDDLE\_SAMPLE}
\end{enumerate}
Had all their values as \textit{Nan}.
\\
After eliminating these 3 columns, we proceeded as before: we eliminated those rows that had \textit{NaN} values and deleted 2 rows.

Then we saved this new data as processed data in its corresponding folder: data/interim/.

%----------------------------------------------------------------------------------------


\labday{Wednesday, 17 October 2018}
\experiment{Feature Selection}

Done by Juan Jos\'e Corroto, Javier C\'ordoba and \'Alvaro Guerrero\\

First, we have modified how we pick the first day data: instead of getting the 2038 first rows, we convert the \textit{'Timestemp'} to days and choosing the day 28 (the first one).\\

In addition to the columns dropped the last week, we have also dropped \textit{UUID}. Then every \textit{FFT} and \textit{Middle Sample} rows have been dropped too.\\

We have chosen just two sensors to do an initial exploration: \textbf{Accelerometer} and \textbf{Linear acceleration}. The purpose is to try and explore the data related to the movement of the phone.\\
So, we create a new dataframe with only \textit{Acelerometer} and \textit{Linear Aceleration}. Finally, we drop \textit{Covariances}, as we don't think they will be useful cause they are calculated as the relation \textit{2-by-2} of the axis.

\experiment{Visualization - Clustering}

We then calculate the \textit{PCA} with an \textit{explained\_variance\_ratio} of \textbf{0.78} and \textbf{0.14}, which is a very good representation of the original 18 features we were studying.

With this results, we plot a \textit{scatterplot} to visualize it(Figure \ref{data plot 1}).

\begin{figure}[h]
\includegraphics[width=0.9\linewidth]{PCA_Plot_Accelerometer_Day1.png}
\setlength\belowcaptionskip{-10pt}
\caption{Plot of the data after PCA}
\label{data plot 1}
\end{figure}



After that we run the \textit{k-means} algorithm with different number of centroids and print the \textit{silhouette} and \textit{distortion}. We initialize the centroids using the \textit{K-means++}algorithm as well. You can see the measurements of distortion and silhouette of KMeans with different number of centroids in figures \ref{K-means distortion 1} and \ref{K-means silhouette 1} respectively.\\


\begin{figure}[h]
\includegraphics[width=0.9\linewidth]{KMeans_Distortion_Accelerometer_Day1.png}
\setlength\belowcaptionskip{-10pt}
\caption{Distortion of K-means with different number of centroids}
\label{K-means distortion 1}
\end{figure}

\begin{figure}[h]
\includegraphics[width=0.9\linewidth]{KMeans_Silhouette_Accelerometer_Day1.png}
\setlength\belowcaptionskip{-10pt}
\caption{Silhouette of K-means with different number of centroids}
\label{K-means silhouette 1}
\end{figure}
By looking at the plots, we have decided to choose \textbf{4} as the final number of centroids.\\

Finally, we run the \textit{k-means} algorithm with said number of centroids and plot the results, each cluster having a different color, and the centroids being colored in red.\\

\begin{figure}[h]
\includegraphics[width=0.9\linewidth]{KMeans_Plot_Accelerometer_Day1.png}
\setlength\belowcaptionskip{-10pt}
\caption{Plot of the data after K-means (centroids in red)}
\label{K-Means plot 1}
\end{figure}


Before trying to get meaning from the clusters, we have decided to repeat this experiment by doing some exploration on the features.


%----------------------------------------------------------------------------------------

\labday{Friday, 26 October 2018}
\experiment{Feature Selection done well}

Done by Juan Jos\'e Corroto.

We have repeated the same process as before: pick the day 1 data from the whole database, but this time, we are going to perform some analysis on the features in order to remove those features that gives us less value due to redundancy, instead of removing some columns randomly.
First, we still remove all columns that have to do with the \textit{Fast Fourier transform}, since we don't know how it works and we can not extract knowledge from them. The same with the \textit{Middle Sample} columns.
After that we remove \textit{NaN} columns and rows as we were doing proviously.\\

Now we are going to see the correlation between the features:

\begin{figure}[h]
\includegraphics[width=0.9\linewidth]{Features_CorrelationMatrix_preDrop_Day1.png}
\setlength\belowcaptionskip{-10pt}
\caption{Correlation of the 24 features we now have}
\label{Correlation predrop}
\end{figure}

We can see some variables with a very high correlation in figure \ref{Correlation predrop}. There are some couple of features with a correlation near to 1 (The maximum value), like features 0 and 1, or 21 and 22. This features are the mean and median values of every axis, so we remove all median values from our features. \\

Having a very similar median and mean values gives us some information: The mean of a value is an stimator strongly affected by outliers, while the median is not. If the median and the mean have similar values, this means little ammount of outliers in our data (It does not mean they are non existent).\\

From the correlation matrix we can also see some features with a very big correlation: the mean values of each axis between both sensors. This means the value of x for the Accelerometer and the LinearAcceleration are highly correlated, so we remove one of those (In this case, the value of the Accelerometer is dropped). The result can be seen in Figure \ref{Correlation postdrop}.

\begin{figure}[h]
\includegraphics[width=0.9\linewidth]{Features_CorrelationMatrix_postDrop_Day1.png}
\setlength\belowcaptionskip{-10pt}
\caption{Correlation of the 15 features we now have after removing some}
\label{Correlation postdrop}
\end{figure}

After removing these features, we are going to use hierarchical clustering to see if we have more redundancies in the values instead of the correlation. As seen in the Figure \ref{Clustering features}, there are little clusters of features, but the most similar ones are features 0, 1, 2 and 12. Those are the variances of the accelerometer. Since those features seem to have very similar values, we can use only one of them.

\begin{figure}[h]
\includegraphics[width=0.9\linewidth]{Features_Dendogram_Day1.png}
\setlength\belowcaptionskip{-10pt}
\caption{Features dendogram}
\label{Clustering features}
\end{figure}

\experiment{Repeating the process}

Done by Juan Jos\'e Corroto.

With our new dataset finished from the previous experiment, we are going to repeat the process of our last day. First we apply PCA with an \textit{explained\_variance\_ratio} of \textbf{0.76} and \textbf{0.14}. This ratio is very similar to the one we had in the previous iteration, and pretty good considering we have removed features that were highly correlated. 
\\

\begin{figure}[h]
\includegraphics[width=0.9\linewidth]{PCA_Plot_Accelerometer_Day1_Selected.png}
\setlength\belowcaptionskip{-10pt}
\caption{Plot of the data after PCA}
\label{data plot 1 Selected}
\end{figure}

The data is plotted in Figure \ref{data plot 1 Selected}, and we can see some differences with respect to the plot we had in the previous day.
\\
We run K-means several times again. We get very good values of Distortion and Silhouette again, thought this time they are even better for 4 clusters, so we select again this number of clusters and run K-means, with the plotted results shown in Figure \ref{K-Means plot 1 Selected}.


\begin{figure}[h]
\includegraphics[width=0.9\linewidth]{KMeans_Distortion_Accelerometer_Day1_Selected.png}
\setlength\belowcaptionskip{-10pt}
\caption{Distortion of K-means with different number of centroids}
\label{K-means distortion 1 Selected}
\end{figure}

\begin{figure}[h]
\includegraphics[width=0.9\linewidth]{KMeans_Silhouette_Accelerometer_Day1_Selected.png}
\setlength\belowcaptionskip{-10pt}
\caption{Silhouette of K-means with different number of centroids}
\label{K-means silhouette 1 Selected}
\end{figure}

\begin{figure}[h]
\includegraphics[width=0.9\linewidth]{KMeans_Plot_Accelerometer_Day1_Selected.png}
\setlength\belowcaptionskip{-10pt}
\caption{Plot of the data after K-means (centroids in red)}
\label{K-Means plot 1 Selected}
\end{figure}


%----------------------------------------------------------------------------------------

\labday{Saturday, 27 October 2018}
\experiment{Using other sensors}

Done by \'Alvaro Guerrero del Pozo.

In this experiment, we have used what we have learnt until now, this time applied to other sensors. Again, we load the dataset, but just keep the data of the first day. Then, we drop any column that contains no information (i.e is null) and rows with null values. Then, we just keep columns related to \textit{GyroscopeStat} and \textit{RotationVector}

As before, we now plot correlations between features:

\begin{figure}[h]
\includegraphics[width=0.9\linewidth]{2710/Features_CorrelationMatrix_preDrop.png}
\setlength\belowcaptionskip{-10pt}
\caption{Correlation of the 18 features}
\label{Correlation Predrop Gyroscope}
\end{figure}


As expected, \textit{Mean} and \textit{Median} values of each axis, of both sensors are highly correlated, so we can safely drop one of them, as the only add redundant informacion. We choose to remove \textit{Median} values. But, there is an exception: X axis of the \textit{Gyroscope}. The correlation between \textit{Mean} and \textit{Median} isn't high enough, so we don't remove any of them.
Also, we can see that there is a surprisingly high (inverse) correlation between \textit{Y Mean} and \textit{X Mean} of the \textit{Rotation Vector}.\\\\\\
It's not as high as other values (most of the previous ones had a correlation of near to 1), but still, with a correlation of around \textbf{-0.75}, we have decided to remove \textit{Y Mean}.

As a result, we are left with 12 features, whose correlations can be seen in the next figure:

\begin{figure}[h]
\includegraphics[width=0.9\linewidth]{2710/Features_CorrelationMatrix_postDrop.png}
\setlength\belowcaptionskip{-10pt}
\caption{Correlation of the 12 features}
\label{Correlation Postdrop Gyroscope}
\end{figure}

\clearpage
After that, we plot the dendogram of the features (Figure \ref{Dendogram Gyroscope}).

\begin{figure}[h]
\includegraphics[width=0.9\linewidth]{2710/Features_Dendogram.png}
\setlength\belowcaptionskip{-10pt}
\caption{Dendogram}
\label{Dendogram Gyroscope}
\end{figure}

As expected, the most similar features are \textit{Mean} and \textit{Median} of the x axis of the gyroscope. We won't remove any of them, as we consider of interest the fact that only in this case, \textit{Mean} and \textit{Median} differ this much. We want this fact to influence the next steps or experiments.

\clearpage
Now that we have our data only with the features that are considered relevant, we proceed to visualize the data, by using \textit{k-means}.\\

First, we apply PCA, and obtain an \textit{explained\_variance\_ratio} of \textbf{0.809} and \textbf{0.128}. It's a high result, so we expect to obtain a good visualization. The plot of PCA is:

\begin{figure}[h]
\includegraphics[width=0.9\linewidth]{2710/PCA_Plot_Gyroscope.png}
\setlength\belowcaptionskip{-10pt}
\caption{Plot of the data after PCA}
\label{Plot Gyroscope}
\end{figure}  

\textit{  }\\\\\\\
After that, we run \textit{k-means} several times, from 2 to 20 \textit{centroids} and plot the Distortion \ref{Distortion Gyroscope} and Silhouette \ref{Silhouettes Gyroscope}.

\begin{figure}[h]
\includegraphics[width=0.9\linewidth]{2710/KMeans_Distortion_Gyroscope.png}
\setlength\belowcaptionskip{-10pt}
\caption{Distortion}
\label{Distortion Gyroscope}
\end{figure}  

\begin{figure}[h]
\includegraphics[width=0.9\linewidth]{2710/KMeans_Silhouttes_Gyroscope.png}
\setlength\belowcaptionskip{-10pt}
\caption{Silhouettes}
\label{Silhouettes Gyroscope}
\end{figure}  

\clearpage

We choose 4 as the best number of \textit{centroids}, as there is a maximun in the \textit{silhouette}, and the \textit{distortion} is low. The resulting plot is shown below.

\begin{figure}[h]
\includegraphics[width=0.9\linewidth]{2710/KMeans_Plot_Gyroscope.png}
\setlength\belowcaptionskip{-10pt}
\caption{Plot of the data after K-means (centroids in red)}
\label{K-Means Gyroscope}
\end{figure}

As we can see, the centroids have been placed where there are higher concentrain of points, but there is a decent amount of points that would be assigned to the neightbour cluster, should the centroids change slightly. Is a matter for another experiment to interpret this results, and decide wether they are good enough or not.

%----------------------------------------------------------------------------------------

\labday{Friday, 26 March 2010}

\experiment{table}

\begin{table}[H]
\begin{tabular}{l l l}
\toprule
\textbf{Groups} & \textbf{Treatment X} & \textbf{Treatment Y} \\
\toprule
1 & 0.2 & 0.8\\
2 & 0.17 & 0.7\\
3 & 0.24 & 0.75\\
4 & 0.68 & 0.3\\
\bottomrule
\end{tabular}
\caption{The effects of treatments X and Y on the four groups studied.}
\label{tab:treatments_xy}
\end{table}

Table \ref{tab:treatments_xy} shows that groups 1-3 reacted similarly to the two treatments but group 4 showed a reversed reaction.

%----------------------------------------------------------------------------------------

\labday{Saturday, 27 March 2010}

\experiment{Bulleted list example} % You don't need to make a \newexperiment if you only plan on referencing it once

This is a bulleted list:

\begin{itemize}
\item Item 1
\item Item 2
\item \ldots and so on
\end{itemize}

%-----------------------------------------

\experiment{example}

\lipsum[6]

%-----------------------------------------

\experiment{example2}

\lipsum[7]

%----------------------------------------------------------------------------------------
%	FORMULAE AND MEDIA RECIPES
%----------------------------------------------------------------------------------------

\labday{} % We don't want a date here so we make the labday blank

\begin{center}
\HRule \\[0.4cm]
{\huge \textbf{Formulae and Media Recipes}}\\[0.4cm] % Heading
\HRule \\[1.5cm]
\end{center}

%----------------------------------------------------------------------------------------
%	MEDIA RECIPES
%----------------------------------------------------------------------------------------

\newpage

\huge \textbf{Media} \\ \\

\normalsize \textbf{Media 1}\\
\begin{table}[H]
\begin{tabular}{l l l}
\toprule
\textbf{Compound} & \textbf{1L} & \textbf{0.5L}\\
\toprule
Compound 1 & 10g & 5g\\
Compound 2 & 20g & 10g\\
\bottomrule
\end{tabular}
\caption{Ingredients in Media 1.}
\label{tab:med1}
\end{table}

%-----------------------------------------

%\textbf{Media 2}\\ \\

%Description

%----------------------------------------------------------------------------------------
%	FORMULAE
%----------------------------------------------------------------------------------------

\newpage

\huge \textbf{Formulae} \\ \\

\normalsize \textbf{Formula 1 - Pythagorean theorem}\\ \\
$a^2 + b^2 = c^2$\\ \\

%-----------------------------------------

%\textbf{Formula X - Description}\\ \\

%Formula

%----------------------------------------------------------------------------------------

\end{document}